\section*{Survey of Literature}
A complex system with individuals interacting with one another according to certain rules, at the macro level, exhibits a totally different function which the sum of all individual functions does not possess (such as structure of time, space, and function), known as emergence, resulting in emergent behavior17. This model can be identified as a major interest in the development of swarm intelligence, as the evolution in functionality promotes a higher degree of adaptability, security, and effectiveness. In this heuristic design space, it is required that the intelligence factor exhibit any dimension of emergent behavior for it be considered swarm intelligence and possess the value as stated in previous sections.  In this section, we outline past and current work towards the development of swarm intelligence that experiences emergent behavior. The literature on these multi-agent systems details numerous tools and algorithms for coordinated motion, formation control, consensus, rendezvous, and flocking which highlights the advancements of swarming behavior.
Abdelkader et al. (2021) presented a summary of the main applications related to aerial swarm systems and the associated research works detailing the main components of any swarm system, localization and mission planning18. The group proposed an abstraction of an aerial swarm system architecture that can help developers to understand the main required modules. For distributed swarm systems with a centralized control architecture that includes a human-in-the-loop (HITL) it is suggested that a structure noted in Fig.X should be used for units with less capability of functionality.  
Fig.
In this regime, various robots carry out their respective tasks locally while coordinating with one another via local communications and sensing. Small multi-UAV systems that lack the individual robot on-board capacity to complete swarm-level mission planning tasks may find utility for this architecture. On the contrary, for decentralized swarm systems, it was proposed that a more distributed model in which there is autonomy in place of the manual operator, allowing the entity’s memory and acquired information transmitted over networks to determine mission success. A detailed representation of this architecture is shown in Fig. X. The latter model sets the platform for emergent behavior within the swarm, which enables the collection of agents to intelligently deduce optimal problem solutions in real-time.
Fig. 
Swarm Intelligence and Metaheuristics
Given that emergent behavior of a swarm derives inspiration from natural social systems, it is important to understand the root of swarming intelligence and how these systems organically evolve towards optimal performance. Much of this can be attributed to metaheuristics, which are algorithms derived from the behavior of social insects, flocks of birds, and fish schools []. Yang et al. derives the relationship between metaheuristics and swarm intelligence, and how these behavioral models have been extensively investigated through the development and application of algorithms that could best represent the possibility self-organization. Certain conditions are necessary for systems to self-organize feedback, stigmergy (when individual system parts intercommunicate indirectly by modifying their local environment), multiple interactions, memory, and environmental setting [18, 20]. These characteristics can be replicated using both deterministic and stochastic algorithms, with each having their own drawbacks and limitations. 
