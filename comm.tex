\section*{Swarm Communication}
Though swarm technology has yet to be practically utilized in commercial applications, there exists great potential. Recent studies indicate that the use of cellular mobile framework can alleviate limiting factors for traditional UAV swarm communication approaches. The use of cellular networks for UAV swarm has the potential to increase swarm efficiency and commercial utility especially with emerging 5G networks with agent-to-agent (A2A) communication capabilities. The implementation of UAVs for 5G requires an efficient distributed learning method to effectively manage the large amount of data generated from the network []. A recent implementation of 5G in swarm applications was tested in September (2022), where Verizon and Lockheed Martin collaborated in flying 5G-enabled drones at the aerospace organization’s test range in Waterton, Colorado over their 5G.MIL pilot network []. The four quadcopter drones used two Verizon’s private network nodes with real-time radio frequency and streaming video data transmitted over 5G millimeter wave links to demonstrate situational awareness, and command and control over all swarm agents. Such a groundbreaking approach sets the premise for further development of swarm communication platforms and with its almost instantaneous information transfer, 5G will allow for agent operations to become almost seamless.