\section*{References}
[1] C. W. Reynolds, “(~) ~ ComputerGraphics, Volume21, Number4, July 1987”.


[2] M. Dorigo et al., “Evolving Self-Organizing Behaviors for a Swarm-Bot,” Autonomous Robots, vol. 17, no. 2/3, pp. 223–245, Sep. 2004, doi: 
10.1023/B:AURO.0000033973.24945.f3.


[3] Z.-Q. Mi and Y. Yang, “Human-Robot Interaction in UVs Swarming: A Survey,” vol. 10, no. 2, 2013.


[4] M. Dorigo, G. Theraulaz, and V. Trianni, “Reflections on the future of swarm robotics,” Sci. Robot., vol. 5, no. 49, p. eabe4385, Dec. 2020, doi: 10.1126/scirobotics.abe4385.


[5] M. Schranz, M. Umlauft, M. Sende, and W. Elmenreich, “Swarm Robotic Behaviors and Current Applications,” Front. Robot. AI, vol. 7, p. 36, Apr. 2020, doi: 10.3389/frobt.2020.00036.


[6] H. V. D. Parunak, “Making Swarming Happen”.


[7] M. Dorigo, G. Theraulaz, and V. Trianni, “Reflections on the future of swarm robotics,” Sci. Robot., vol. 5, no. 49, p. eabe4385, Dec. 2020, doi: 10.1126/scirobotics.abe4385.


[8] R. Arnold, K. Carey, B. Abruzzo, and C. Korpela, “What is A Robot Swarm: A Definition for Swarming Robotics,” in 2019 IEEE 10th Annual Ubiquitous Computing, Electronics & Mobile Communication Conference (UEMCON), New York City, NY, USA: IEEE, Oct. 2019, pp. 0074–0081. doi: 10.1109/UEMCON47517.2019.8993024.


[9] A. Haider, A. Schmidt, “Defining the Swarm: Challenges in Developing NATO-Agreed Terminology Across All Domains” JAPCC, Journal Edition 34, 2022, pg. 60 – 65.


[10] R. Bridley, S. Pastor, “Military Drone Swarms, and the Options to Combat Them” Small Wars Journal, 2022.


[11] P. R. Padave and J. S. Reddy, “Effectiveness of Swarm Robotics in Coordination of Multiple Robots as a System That Consists of Large Numbers of Mostly Simple Physical Robots,” TTIRAS, vol. 2, no. 3, Sep. 2022, doi: 10.36647/TTIRAS/02.03.A002.


[12] J. Penders, “Robot Swarming Applications”.


[13] S. J. A. Edwards, Swarming on the battlefield: past, present, and future. in MR / Rand OSD, no. 1100. Santa Monica, CA: Rand, 2000.


[14] M. S. Abbasi, H. Al-Sahaf, M. Mansoori, and I. Welch, “Behavior-based ransomware classification: A particle swarm optimization wrapper-based approach for feature selection,” Applied Soft Computing, vol. 121, p. 108744, May 2022, doi: 10.1016/j.asoc.2022.108744.


[15] V. Strobel, A. Pacheco, and M. Dorigo, “Robot swarms neutralize harmful Byzantine robots using a blockchain-based token economy,” Sci. Robot., vol. 8, no. 79, p. eabm4636, Jun. 2023, doi: 10.1126/scirobotics.abm4636.


[16] G. Bansal, V. Chamola, B. Sikdar, and F. R. Yu, “UAV SECaaS: Game-Theoretic Formulation for Security as a Service in UAV Swarms,” IEEE Systems Journal, vol. 16, no. 4, pp. 6209–6218, Dec. 2022, doi: 10.1109/JSYST.2021.3116213.


[17] Q. Liu, M. He, D. Xu, N. Ding, and Y. Wang, “A Mechanism for Recognizing and Suppressing the Emergent Behavior of UAV Swarm,” Mathematical Problems in Engineering, vol. 2018, pp. 1–14, Sep. 2018, doi: 10.1155/2018/6734923.


[18] M. Abdelkader, S. Güler, H. Jaleel, and J. S. Shamma, “Aerial Swarms: Recent Applications and Challenges,” Curr Robot Rep, vol. 2, no. 3, pp. 309–320, Sep. 2021, doi: 10.1007/s43154-021-00063-4.


[19] X.-S. Yang, S. Deb, S. Fong, X. He, and Y.-X. Zhao, “From Swarm Intelligence to Metaheuristics: Nature-Inspired Optimization Algorithms,” Computer, vol. 49, no. 9, pp. 52–59, Sep. 2016, doi: 10.1109/MC.2016.292.


[20] M. Mannone, V. Seidita, and A. Chella, “Modeling and designing a robotic swarm: A quantum computing approach,” Swarm and Evolutionary Computation, vol. 79, p. 101297, Jun. 2023, doi: 10.1016/j.swevo.2023.101297.


[21] D. Xing, Z. Zhen, and H. Gong, “Offense–defense confrontation decision making for dynamic UAV swarm versus UAV swarm,” Proceedings of the Institution of Mechanical Engineers, Part G: Journal of Aerospace Engineering, vol. 233, no. 15, pp. 5689–5702, Dec. 2019, doi: 10.1177/0954410019853982.


[22] M. Dorigo, G. Theraulaz, and V. Trianni, “Reflections on the future of swarm robotics,” Sci. Robot., vol. 5, no. 49, p. eabe4385, Dec. 2020, doi: 10.1126/scirobotics.abe4385.


[23] M. Schranz et al., “Swarm Intelligence and cyber-physical systems: Concepts, challenges and future trends,” Swarm and Evolutionary Computation, vol. 60, p. 100762, Feb. 2021, doi: 10.1016/j.swevo.2020.100762.


[24] A. Srivastava, “Swarm Intelligence for Network Security: A New Approach to User Behavior Analysis,” vol. 10, no. 02, 2023.


[25] M. Dorigo, G. Theraulaz, and V. Trianni, “Swarm Robotics: Past, Present, and Future [Point of View],” Proc. IEEE, vol. 109, no. 7, pp. 1152–1165, Jul. 2021, doi: 10.1109/JPROC.2021.3072740.


[26] L. Xie, T. Han, H. Zhou, Z.-R. Zhang, B. Han, and A. Tang, “Tuna Swarm Optimization: A Novel Swarm-Based Metaheuristic Algorithm for Global Optimization,” Computational Intelligence and Neuroscience, vol. 2021, pp. 1–22, Oct. 2021, doi: 10.1155/2021/9210050.


[27] Y. Su, H. Zhou, and Y. Deng, “D2D-Based Cellular-Connected UAV Swarm Control Optimization via Graph-Aware DRL,” in GLOBECOM 2022 - 2022 IEEE Global Communications Conference, Rio de Janeiro, Brazil: IEEE, Dec. 2022, pp. 1326–1331. doi: 10.1109/GLOBECOM48099.2022.10001506.


[28] Q. Ouyang, Z. Wu, Y. Cong, and Z. Wang, “Formation control of unmanned aerial vehicle swarms: A comprehensive review,” Asian Journal of Control, vol. 25, no. 1, pp. 570–593, Jan. 2023, doi: 10.1002/asjc.2806.


[29] L. Xiang and T. Xie, “Research on UAV Swarm Confrontation Task Based on MADDPG Algorithm,” in 2020 5th International Conference on Mechanical, Control and Computer Engineering (ICMCCE), Harbin, China: IEEE, Dec. 2020, pp. 1513–1518. doi: 10.1109/ICMCCE51767.2020.00332.


[30] X. Jin, Z. Wang, J. Zhao, and D. Yu, “Swarm control for large-scale omnidirectional mobile robots within incremental behavior,” Information Sciences, vol. 614, pp. 35–50, Oct. 2022, doi: 10.1016/j.ins.2022.09.061.


[31] B. Wang, S. Li, X. Gao, and T. Xie, “UAV Swarm Confrontation Using Hierarchical Multiagent Reinforcement Learning,” International Journal of Aerospace Engineering, vol. 2021, pp. 1–12, Dec. 2021, doi: 10.1155/2021/3360116.


[32] W. He, H. Yao, F. Wang, Z. Wang, and Z. Xiong, “Enhancing the Efficiency of UAV Swarms Communication in 5G Networks through a Hybrid Split and Federated Learning Approach,” in 2023 International Wireless Communications and Mobile Computing (IWCMC), Marrakesh, Morocco: IEEE, Jun. 2023, pp. 1371–1376. doi: 10.1109/IWCMC58020.2023.10183145.


[33] Á. Gutiérrez, “Recent Advances in Swarm Robotics Coordination: Communication and Memory Challenges,” Applied Sciences, vol. 12, no. 21, p. 11116, Nov. 2022, doi: 10.3390/app122111116.


[34] M. Campion, P. Ranganathan, and S. Faruque, “UAV swarm communication and control architectures: a review,” J. Unmanned Veh. Sys., vol. 7, no. 2, pp. 93–106, Jun. 2019, doi: 10.1139/juvs-2018-0009.


[35] D. P. May, “28-Drone Swarm Just Led The Way For A Simulated Air Assault Mission”.


[36] S. Hai-wen, Y. Shao-zhen, Z. Mo, M. Xiang-yao, L. Dan, and T. Jia-yu, “Anti UAV swarm warfare command and control system”.


[37] A. B. Taylor, “COUNTER-UNMANNED AERIAL VEHICLES STUDY: SHIPBOARD LASER WEAPON SYSTEM ENGAGEMENT STRATEGIES FOR COUNTERING DRONE SWARM THREATS IN THE MARITIME ENVIRONMENT”.


[38] W. Chen, X. Meng, J. Liu, H. Guo, and B. Mao, “Countering Large-Scale Drone Swarm Attack by Efficient Splitting,” IEEE Trans. Veh. Technol., vol. 71, no. 9, pp. 9967–9979, Sep. 2022, doi: 10.1109/TVT.2022.3178821.


[39] X. Shaohui et al., “Development of a shooting strategy to neutralize UAV swarms based on multi-shot cooperation,” J. Phys.: Conf. Ser., vol. 2460, no. 1, p. 012035, Apr. 2023, doi: 10.1088/1742-6596/2460/1/012035.


[40] M. J. Guitton, “Fighting the Locusts: Implementing Military Countermeasures Against Drones and Drone Swarms,” Scandinavian Journal of Military Studies, vol. 4, no. 1, pp. 26–36, Jan. 2021, doi: 10.31374/sjms.53.


[41] J. Simonjan, S. R. Probst, and M. Schranz, “Inducing Defenders to Mislead an Attacking UAV Swarm,” in 2022 IEEE 42nd International Conference on Distributed Computing Systems Workshops (ICDCSW), Bologna, Italy: IEEE, Jul. 2022, pp. 278–283. doi: 10.1109/ICDCSW56584.2022.00059.


[42] S. Park, H. T. Kim, S. Lee, H. Joo, and H. Kim, “Survey on Anti-Drone Systems: Components, Designs, and Challenges,” IEEE Access, vol. 9, pp. 42635–42659, 2021, doi: 10.1109/ACCESS.2021.3065926.


[43] G. V. Weinberg, “Prediction of UAV Swarm Defeat With High-Power Radio Frequency Fields,” IEEE Trans. Electromagn. Compat., vol. 64, no. 6, pp. 2157–2162, Dec. 2022, doi: 10.1109/TEMC.2022.3193881.


[44] J. Yan, H. Xie, and D. Zhuang, “The Adaptability of Countermeasures Against UAV Swarms in Typical Mission Scenarios,” in Proceedings of 2021 International Conference on Autonomous Unmanned Systems (ICAUS 2021), M. Wu, Y. Niu, M. Gu, and J. Cheng, Eds., in Lecture Notes in Electrical Engineering, vol. 861. Singapore: Springer Singapore, 2022, pp. 2436–2447. doi: 10.1007/978-981-16-9492-9_240.


[45] C. Conte, S. Verini Supplizi, G. De Alteriis, A. Mele, G. Rufino, and D. Accardo, “Using Drone Swarms as a Countermeasure of Radar Detection,” Journal of Aerospace Information Systems, vol. 20, no. 2, pp. 70–80, Feb. 2023, doi: 10.2514/1.I011131.


[46] T. Chung and R. Daniel, “DARPA OFFSET: A Vision for Advanced Swarm Systems through Agile Technology Development and ExperimentationDARPA OFFSET: A Vision for Advanced Swarm Systems through Agile Technology Development and Experimentation,” FR, vol. 3, no. 1, pp. 97–124, Jan. 2023, doi: 10.55417/fr.2023003.


[47] I. U. Din, K. A. Awan, A. Almogren, and J. J. P. C. Rodrigues, “Swarmtrust: A swarm optimization-based approach to enhance trustworthiness in smart homes,” Physical Communication, vol. 58, p. 102064, Jun. 2023, doi: 10.1016/j.phycom.2023.102064.


[48] C. Qu et al., “UAV Swarms in Smart Agriculture: Experiences and Opportunities,” in 2022 IEEE 18th International Conference on e-Science (e-Science), Salt Lake City, UT, USA: IEEE, Oct. 2022, pp. 148–158. doi: 10.1109/eScience55777.2022.00029.