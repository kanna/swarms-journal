\section*{Abstract}
The deployment of large groups of robot swarms which coordinates and cooperatively solves a problem remains an engineering challenge. In recent years, the tactic of employing UAV swarms to break through enemy defense systems, carry out a saturated attack, and intercept the intruders will be the trends of UAV applications. This potential calls for an appropriate definition of characteristics, behavior, and ensured countermeasures to be available.Swarm robotics takes inspiration from natural self-organizing systems such as social insects fish schools, or bird flocks. The definition can be loosely defined as a collection of actors congregated together for a specific mission or function. A swarm can also be defined as a formation of multiple entities, which display coordinated behavior towards an objective. The challenge with defining ‘swarm’ is that the applicable uses differ significantly and that the defining parameters for one use may not be relevant to another. \kc{Do you think this is expansive enough?}


\section*{Introduction}

\section*{Definition of Swarming}

\section*{Value of Swarming}
\subsection*{Resilience of Swarms}