\section*{Swarm Countermeasures}

The use of swarms been demonstrated in a variety of scenarios, largely within laboratory settings or until recently, within visual demonstrations for public outreach and entertainment []. There have been few cases in which swarms have been employed within military tactics, as the overall technological infrastructure and value is still in question. However, the deployment of large-scale mass attacks using drones have been seen in recent conflicts between Russia and Ukraine [] (2023). In their ongoing effort for increasing defense and surveillance effectiveness, Ukraine launched an attack on the Sevastopol Naval Base housing Russia’s Black Sea Fleet with drones and kamikaze “drone boats”, which were unmanned surface vessels filled with explosives. Surveillance drones were also implemented to strike Engels Airbase, hundreds of kilometers beyond Russian borders []. Although these units were deployed in large numbers, they still lacked the technological sophistication to be truly considered a swarm. Therefore, existing technology towards countering mass attacks should be applied and fitted towards a swarm in congruence with advancing swarm technologies and features. 
Currently, there exists several options for potentially countering swarms based on mass attack scenarios. Technology such as microwaves, lasers, jamming systems, underground buildings, kinetic interceptors, and defensive swarms, to name a few, are seen as prominent candidates as effective weaponry against these threats []. The most well-known example of this type of capability is the Air Force's use of Tactical High-power Operational Responder, or THOR, a short-range directed energy weapon created by the Air Force Research Laboratory, to battle swarms []. THOR drops drones from the sky by interfering with their electronics using microwaves. With its initial release in 2019, the system has undergone extensive development. THOR was the first to successfully take down a large group of drones, and it happened earlier this year (2023). Another example of an emerging countermeasure is Lockheed Martin’s new ground-based system (ICARUS), a system which uses radio frequencies to counter invading mass attacks. ICARUS’ key features use acoustic and passive imaging sensors to acquire target information and perform target recognition and tracking. It can then conduct target threat judgment and task assignment. The system can also destroy drone swarm targets via cyber electromagnetic frequencies []. 
